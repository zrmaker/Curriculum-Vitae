\documentclass[letterpaper,9pt]{article}

\usepackage{tabularx}
\usepackage{hyperref}
\usepackage{geometry}
\usepackage{fancyhdr}
\usepackage[T1]{fontenc}
%\usepackage[sc,osf]{mathpazo}
\usepackage[sc]{mathpazo}
\usepackage{sectsty}

\pagestyle{fancy}
\PassOptionsToPackage{hyphens}{url}\usepackage{hyperref}

\def\name{Renyuan Zhang}

\hypersetup{
  colorlinks = true,
  urlcolor = black,
  pdfauthor = {\name},
  pdfkeywords = {Renyuan Zhang, Curriculum Vitae, Resume},
  pdftitle = {\name: Curriculum Vitae},
  pdfsubject = {Curriculum Vitae},
  %pdfpagemode = UseNone
}

\geometry{
%  body={6.5in, 9in},
  left=.75in,
  top=.75in,
  right=.75in,
  bottom=1in
}

\linespread{1.0}
\setlength{\parskip}{-5pt}
\setlength{\parindent}{0pt}

\urlstyle{rm}
\fancyhf{}
\rfoot{\thepage}
\pagestyle{fancy}
\renewcommand{\headrulewidth}{0pt}

\renewcommand{\familydefault}{\rmdefault}
\sectionfont{\rmfamily\mdseries\large}
\subsectionfont{\rmfamily\mdseries\itshape\small}
\subsubsectionfont{\rmfamily\mdseries\itshape}
\newcommand{\etal}{\textit{et al}.}
\newcommand{\ie}{\textit{i}.\textit{e}.}
\newcommand{\eg}{\textit{e}.\textit{g}.}

% Other possible font commands include:
% \ttfamily for teletype,
% \sffamily for sans serif,
% \bfseries for bold,
% \scshape for small caps,
% \normalsize, \large, \Large, \LARGE sizes.

% Make lists without bullets
\renewenvironment{itemize}{
  \begin{list}{}{
    \setlength{\topsep}{0pt}
    \setlength{\itemsep}{0pt}
    \setlength{\parsep}{0pt}
    \setlength{\partopsep}{0pt}
    \setlength{\leftmargin}{1.5em}
  }
}{\end{list}}

\begin{document}
{\huge \name \textnormal \ (Leo), Ph.D.}

\vspace{1em}

\begin{minipage}{0.55\linewidth}
  Ph.D. in Electrical and Computer Engineering \\
  % Research Assistant \\
  % \href{http://ece.arizona.edu/}{Department of Electrical and Computer Engineering} \\
  \href{http://www.arizona.edu/}{\it University of Arizona} \\
  Tucson, AZ 85721
\end{minipage}
\begin{minipage}{0.4\linewidth}
  \begin{tabular}{ll}
    Phone: & (520) 878-8630 \\
    Email: & \href{mailto:ryzhang@email.arizona.edu}{ryzhang@email.arizona.edu} \\
    % Address: & 2724 N Neruda Ln, Tucson, AZ 85712\\
    LinkedIn: & \href{https://www.linkedin.com/in/zrmaker/}{https://www.linkedin.com/in/zrmaker/} \\
    GitHub: & \href{https://github.com/zrmaker}{zrmaker}
  \end{tabular}
\end{minipage}

%\section*{Personal}
%
%\begin{itemize}
%\item Born on November 30, 1990.
%\item Suzhou, China.
%\end{itemize}

\section*{Objective}
\begin{itemize}
  % \setlength{\itemindent}{1em}
  % \setlength{\itemsep}{.25em}
  \item Seeking full time opportunity in automotive radar, radar signal processing and autonomous driving, expected to graduate on May 2019.
\end{itemize}

\section*{Education}

\begin{itemize}
  \setlength{\itemsep}{1em}
  \item Ph.D. in Electrical and Computer Engineering\hfill Aug. 2015 - Present\\ {\it University of Arizona}\\Research interest: radar signal processing, micro-Doppler signatures, sensor fusion, non-coherent integration, machine learning, synthetic aperture radar.
  \item M.S. in Optical Sciences\hfill Aug. 2013 - Aug. 2015\\ {\it University of Arizona}\\Research interest: optical imaging, medical imaging, optical coherence tomography.
  \item B.S. in Optoelectronic Engineering\hfill Sept. 2009 - June 2013\\ {\it Chongqing University}
  \item Udacity Self-Driving Car Nanodegree\hfill Oct. 2017 - Sep. 2018\\ {\it Udacity}
\end{itemize}


\section*{Professional Experience}
\begin{itemize}
  \setlength{\itemsep}{1em}
  \item Research Assistant at Department of Electrical and Computer Engineering \hfill 2015 - Present\\
Advisor: Dr. Siyang Cao, {\it University of Arizona}.
  \begin{itemize}
    \item [$\bullet$] Developing multi-target multi-input camera-radar fusion and classification.
    \item [$\bullet$] Researching non-synchronized incoherent MIMO radar angle resolution improvements.
    \item [$\bullet$] Researching radar point cloud clustering and detection by machine learning method.
    \item [$\bullet$] Developing CUDA algorithms on radar signal processing.
    \item [$\bullet$] Researching on radar target clustering and classification.
    \item [$\bullet$] Achieved human behavior detection via CNN using micro-Doppler signatures by mmWave radar.
    % Fields: Micro-Doppler Signature, CNN.\\
    % A real-time, accurate and portable mmWave radar behavior detecting system is achieved.
    % The detection and classification is based on micro-Doppler signature of the radar targets. A convolution neural network (CNN) is implemented in the human behavior detection.
    \item [$\bullet$] Realized radar interference detection, classification and mitigation using SVM.
    % Fields: Radar Interference (RFI), Machine Learning, DSP.\\
    % Working on simulation of 77 GHz automotive radar interference. Use machine learning methods to classify different interference range-doppler response results from PRI difference, long chirp, etc.. And use filter design, circular scanning array antenna and advanced signal processing methods to mitigate interference.

    % \item [--] Automotive Radar Measurement with 3D Printing Lens\\
    % With Min Liang and Jin-pil Tak.\\
    % Fields: Automotive Radar, DSP, RF and Antenna Theory, Antenna Measurements.\\
    % Measurements of real automobile with 77 GHz automotive radar and 3D priting lens (URL: \url{http://techlaunch.arizona.edu/news/startup-licenses-ua-invented-radar-system}).
    \item [$\bullet$] Completed 3D imaging millimeter wave circular SAR using single transceiver and compressed sensing.
  \end{itemize}

  % \item Freelance Engineer at Hertzwell \hfill June - July 2018\\
  % {\it Hertzwell\textsuperscript{\textregistered}}, Singapore.
  % \begin{itemize}
  %   \item [$\bullet$] Developed and evaluated Inras\textsuperscript{\textregistered} 77 GHz automotive radar with CFAR and clustering.
  %   \item [$\bullet$] Developed TI\textsuperscript{\textregistered} 77 GHz AWR1443 evaluation board with multi-sensor support and sensor fusion support.
  %   \item [$\bullet$] Developed TI\textsuperscript{\textregistered} radar-CV camera overlay.
  % \end{itemize}

  \item Sensor Engineer Intern at TuSimple \hfill Sept. 2017 - Mar. 2018\\
  {\it TuSimple LLC\textsuperscript{\textregistered}}, Tucson, AZ.
  \begin{itemize}
    \item [$\bullet$] Developed and evaluated Autoliv\textsuperscript{\textregistered} 77 GHz multi-mode radar ROS driver.
    \item [$\bullet$] Developed Bosch\textsuperscript{\textregistered} 77 GHz long-range radar and mid-range radar ROS driver.
    \item [$\bullet$] Evaluated Delphi\textsuperscript{\textregistered} 77 GHz electronic scanning radar.
    \item [$\bullet$] Finished Hokuyo\textsuperscript{\textregistered} URG-04LX-UG01 Scanning Laser Rangefinder development and truck trailer monitor/filter project.
    \item [$\bullet$] Written industrial radar signal filtering and target recognition.

  \end{itemize}

  \item Research Assistant of Nonlinear Optics at College of Optical Sciences \hfill 2014 - 2015\\
  Advisor: Dr. Khanh Kieu, {\it University of Arizona}.
  \begin{itemize}
    \item [$\bullet$] Developed and analyzed with hospitals using fiber based SD-OCT.
    % Fields: Optical Imaging, Interference, Lens Design, Spectral Domain Analysis, Fiber Optics, Medical Imaging, OCT.
  \end{itemize}


  % \item Research Assistant of Applied Optics at College of Optical Sciences \hfill 2013 - 2014\\
  % Advisor: Dr. Rongguang Liang, {\it University of Arizona}.
  % \begin{itemize}
  % \item [$\bullet$] Confocal Microscopy
  % % Fields: Optical Imaging, Spatial Pinhole, Lens Design.
  % \end{itemize}


\end{itemize}




\section*{Skills}
\begin{tabularx}{\textwidth}{lX}
  {\bf Programming:} & Python, Mathworks\textsuperscript{\textregistered} MATLAB, R, NI LabVIEW, C/C++/C\#, JAVA.\\
  {\bf Sensors:} & Radar, LiDAR, CMOS camera, CCD camera, sonar, microphone and Microsoft\textsuperscript{\textregistered} Kinect.\\
  {\bf Machine Learning:} & SVM, ANN, CNN, RNN, {\it k}-NN, {\it k}-means, naive Bayes, decision tree and mixture model (Gaussian).\\
  {\bf CAD \& Production:} & SOLIDWORKS, Autodesk\textsuperscript{\textregistered} AutoCAD and Adobe\textsuperscript{\textregistered} Creative Cloud (Photoshop, Illustrator, Premiere Pro).\\
  {\bf RF \& EM:} & ANSYS\textsuperscript{\textregistered} EM suite and Keysight\textsuperscript{\textregistered} ADS.\\
  {\bf Operating Systems:} & Windows and Ubuntu.\\
  {\bf Embedded Systems:} & NI\textsuperscript{\textregistered} control and acquisition suites and Arduino.\\
  {\bf Others:} & Digital signal processing (DSP), imaging processing, Nvidia\textsuperscript{\textregistered} CUDA, source control (git) and controller area network (CAN).\\
\end{tabularx}



\section*{Projects}
\begin{itemize}
  \setlength{\itemsep}{1em}
  \item Non-synchronized integration using multiple FMCW MIMO radars
  \begin{itemize}
    \item [$\bullet$] In this project, the non-synchronized integration of multiple frequency-modulated continuous-wave (FMCW) radars is presented. A phase error deduction method on different nonsynchronized radars using trust-region-reflective least squares algorithm is introduced. Better angle of arrival (AoA) estimation,
    better angular resolution and better side lobes deduction are realized in experimental result. Therefore, integrating multiple independent radar systems to emulate a large aperture is achieved.
    \item [$\bullet$] Paper: \\
    {\bf R. Zhang} and S. Cao, "Non-Synchronized Integration using Multiple Radars via Least Squares Fitting," {\it 2019 IEEE Radar Conference}. (submitted Oct. 2018)
  mmWave radar.
  \end{itemize}

  \item Human behavior detection and classification using micro-Doppler signature
  \begin{itemize}
    \item [$\bullet$] A real-time behavior detection system using millimeter wave (mmWave) radar is presented in this project. Radar is used to sense the micro-Doppler information of targets. A convolution neural network (CNN) is further implemented in the detecting and classifying the human motion behaviors using this information. Both the convolution layers and architecture of CNNs are presented. The analysis on loss and accuracy of training results are also shown. The experimental result indicates a precise determination of human motion behavior detection using the proposed system.
    \item [$\bullet$] Paper: \\
    \href{https://ieeexplore.ieee.org/document/8585077}{{\bf R. Zhang} and S. Cao, "Real-time Human Motion Behavior Detection via CNN using mmWave Radar," {\it IEEE Sensors Letters}.}\\
    F. Jin, {\bf R. Zhang} \etal, "Multiple Patients Behavior Detection in Real-time using mmWave Radar and Deep CNNs," {\it 2019 IEEE Radar Conference}. (submitted Oct. 2018)
  \end{itemize}

  \item Radar point cloud clustering and machine learning
  \begin{itemize}
    \item [$\bullet$] In this project, a robust and adaptive radar point cloud clustering algorithm, named radar elliptical density-based spatial clustering of applications with noise (REDBSCAN), is presented. The proposed algorithm shows a better clustering results for adapting to arbitrary shape of targets as well as any number of targets comparing with traditional clustering methods. The algorithm is presented, and is implemented in experiments using the state-of-art millimeter wave (mmWave) radar sensor with multiple-input multiple-output (MIMO) antennas. The related signal processing chain and the clustering outcomes are also discussed.
    \item [$\bullet$] Paper: \\
    {\bf R. Zhang} and S. Cao, "Robust and Adaptive Radar Elliptical Density-Based Spatial Clustering and Labelling for mmWave Radar Point Cloud Data," {\it 2019 IEEE Radar Conference}. (submitted Oct. 2018)
  \end{itemize}

  

  \item Automotive radar interference detection, classificationa and mitigation
  \begin{itemize}
    \item [$\bullet$] The project studies classification of the automotive radar interference waveforms via support vector machine (SVM). Automotive radar implemented in advanced driver assistance systems (ADASs) is an essential sensor in road traffic safety, e.g., moving target indication, collision avoidance and enhanced navigation system. However, radar-to-radar interference is inevitable as the number of automotive radar increases. Our work shows different types of radar-to-radar interference with analyzing the received signal. Providing linear frequency modulated transmitting (LFM) signal, filtering and dechirping techniques, the classification of six different types of radar-to-radar interference are presented and analyzed. Time-frequency domain signal and range-doppler profiles of different types of interference are simulated. The machine learning classifier of SVM of multi-class high-dimensional waveform data classification is used to classify different interference waveforms and noninterference waveforms. Random produced dataset is cross validated through the SVM classifier. Different types of interference prediction accuracies are shown and verified by the proposed method. The confusion matrix of interference and noninterference, detecting different types of interference and classifying all incoming receiving signal are presented.
    \item [$\bullet$] Paper: \\
    \href{https://ieeexplore.ieee.org/abstract/document/8378586/}{{\bf R. Zhang} and S. Cao, "Support vector machines for classification of automotive radar interference," {\it 2018 IEEE Radar Conference (RadarConf18)}, Oklahoma City, OK, 2018, pp. 0366-0371.}
  \end{itemize}

  \item 3D imaging millimeter wave circular synthetic aperture radar
  \begin{itemize}
    \item [$\bullet$] In this project, a newmillimeter wave 3D imaging radar is proposed. The user just needs to move the radar along a circular track, a high resolution 3D imaging can be generated. The proposed radar uses the movement of itself to synthesize a large aperture in both the azimuth and elevation directions. It can utilize inverse Radon transform to resolve 3D imaging. To improve the sensing result, compressed sensing approach is further investigated. The simulation and experimental result further illustrated the design. Because a single transceiver circuit is needed, a light, affordable and high resolution 3D mmWave imaging radar is illustrated.
    \item [$\bullet$] Paper: \\
    \href{http://www.mdpi.com/1424-8220/17/6/1419}{{\bf R. Zhang} and S. Cao, "3D Imaging Millimeter Wave Circular Synthetic Aperture Radar," {\it Sensors}, vol. 17, no. 6, p. 1419, June 2017.} \\
    \href{http://ieeexplore.ieee.org/document/7944286/}{{\bf R. Zhang} and S. Cao, "Compressed Sensing For Portable Millimeter Wave 3D Imaging Radar," {\it 2017 IEEE Radar Conference (RadarConf)}, Seattle, WA, USA, May 2017, pp. 0663-0668.} \\
    \href{http://ieeexplore.ieee.org/abstract/document/7944216/}{{\bf R. Zhang} and S. Cao, "Portable Millimeter Wave 3D Imaging Radar," {\it 2017 IEEE Radar Conference (RadarConf)}, Seattle, WA, USA, May 2017, pp. 0298-0303.}
  \end{itemize}

  \item Optical coherence tomography
  \begin{itemize}
    \item [$\bullet$] Spectral Domain Optical Coherence Tomography (SD-OCT) is introduced in this project. In comparison to the first generation Time Domain Optical Coherence Tomography (TDOCT), SD-OCT is superior in terms of its capturing speed, signal to noise ratio, and sensitivity. The SD-OCT has been widely used in both clinical and research imaging. The primary goal of this research is to design and construct a Spectral Domain Optical Coherence Tomography system which consists of a fiber-based imaging system and a line scan CCD-based high-speed spectrometer, and is capable of imaging and analyzing biological tissue at a wavelength of 1040 nm. Additionally, a NI LabVIEW software for controlling, acquiring and signal processing is developed and implemented. An axial resolution of 16.9 micrometer is achieved, and 2-D greyscale images of various samples have been obtained from our SD-OCT system. The device was initially calibrated using a glass coverslip, and then tested on multiple biological samples, including the distal end of a human fingernail, onion peels, and pancreatic tissues. In each of these images, both tissue and cell structures were observed at depths of up to 0.6 millimeter. The A-scan processing time is 8.445 millisecond. Our SD-OCT system demonstrates tremendous potential in becoming a vital imaging tool for clinicians and researchers.
    \item [$\bullet$] Thesis: \\
    \href{https://scholar.google.com/citations?view_op=view_citation&hl=en&user=PpPf3BoAAAAJ&citation_for_view=PpPf3BoAAAAJ:9yKSN-GCB0IC}{{\bf R. Zhang} and K. Kieu, "Fiber Based Spectral Domain Optical Coherence Tomography: Mechanism and Clinical Applications," {\it University of Arizona}, 2015.}
  \end{itemize}

  \item Multi-target multi-input camera-radar sensor fusion and classification
  \begin{itemize}
    \item [$\bullet$] In this project, comprehensive sensor fusion by automotive radar and camera is studied. Motion based multi-target tracking and classification is achieved. Implemented micro-Doppler signature machine learning, region CNN and YOLO allows the target recognition and classification. In this project, the better target detection and tracking and the better target recognition and classification using dual sensors are realized.
    \item [$\bullet$] Paper in progress.
  \end{itemize}

  \item CUDA on radar signal processing
  \begin{itemize}
    \item [$\bullet$] Using CUDA libray to optimize radar signal processing chain is fully researched in this project. Improved automotive radar algorithms are coded via C and Python. The time cost for signal processing is largely reduced and the mass processing to obtain range-Doppler-azimuth response and STAP is greatly improved in this project.
  \end{itemize}

  \item Smart antenna on automotive radars
  \begin{itemize}
    \item [$\bullet$] In this project, a circular smart antenna for automotive radar is designed. The smart antenna can enhance the receiving power from designated directions. The antenna pattern and array factors are presented.
  \end{itemize}

  \item Automotive radar SAR terrain mapping
  \begin{itemize}
    \item [$\bullet$] The synthetic aperture terrain mapping by emulating multiple apertures over automotive patch antenna is achieved in this project. The automotive radar is fused with GPS and IMU to get the ground truth. The automotive radar terrain mapping is conducted on autonomous driving test vehicle of ECE department of University of Arizona.
  \end{itemize}
\end{itemize}


\section*{Publications}

% \subsection*{US Patents}

% \begin{itemize}
%   \item F. Deng, {\bf R. Zhang} and L. Nie, "Truck Trailer Angle Measurement using Single Beam Lidar," {\it US Patents}. (submitted 2018)

% \end{itemize}

\subsection*{Journal Articles}

\begin{itemize}
  \item \href{https://ieeexplore.ieee.org/document/8585077}{{\bf R. Zhang} and S. Cao, "Real-time Human Motion Behavior Detection via CNN using mmWave Radar," {\it IEEE Sensors Letters}.}
  \item \href{http://www.mdpi.com/1424-8220/17/6/1419}{{\bf R. Zhang} and S. Cao, "3D Imaging Millimeter Wave Circular Synthetic Aperture Radar," {\it Sensors}, vol. 17, no. 6, p. 1419, June 2017.}
\end{itemize}

\subsection*{Proceedings}

\begin{itemize}
  \item {\bf R. Zhang} and S. Cao, "Non-Synchronized Integration using Multiple Radars via Least Squares Fitting," {\it 2019 IEEE Radar Conference}. (submitted Oct. 2018)
  \item {\bf R. Zhang} and S. Cao, "Robust and Adaptive Radar Elliptical Density-Based Spatial Clustering and Labelling for mmWave Radar Point Cloud Data," {\it 2019 IEEE Radar Conference}. (submitted Oct. 2018)
  \item F. Jin, {\bf R. Zhang} \etal, "Multiple Patients Behavior Detection in Real-time using mmWave Radar and Deep CNNs," {\it 2019 IEEE Radar Conference}. (submitted Oct. 2018)
  \item \href{https://ieeexplore.ieee.org/abstract/document/8378586/}{{\bf R. Zhang} and S. Cao, "Support vector machines for classification of automotive radar interference," {\it 2018 IEEE Radar Conference (RadarConf18)}, Oklahoma City, OK, 2018, pp. 0366-0371.}
  \item \href{http://ieeexplore.ieee.org/document/7944286/}{{\bf R. Zhang} and S. Cao, "Compressed Sensing For Portable Millimeter Wave 3D Imaging Radar," {\it 2017 IEEE Radar Conference (RadarConf)}, Seattle, WA, USA, May 2017, pp. 0663-0668.}
  \item \href{http://ieeexplore.ieee.org/abstract/document/7944216/}{{\bf R. Zhang} and S. Cao, "Portable Millimeter Wave 3D Imaging Radar," {\it 2017 IEEE Radar Conference (RadarConf)}, Seattle, WA, USA, May 2017, pp. 0298-0303.}
\end{itemize}

\subsection*{Dissertation and Thesis}

\begin{itemize}
  \item \href{https://scholar.google.com/citations?view_op=view_citation&hl=en&user=PpPf3BoAAAAJ&citation_for_view=PpPf3BoAAAAJ:9yKSN-GCB0IC}{{\bf R. Zhang} and K. Kieu, "Fiber Based Spectral Domain Optical Coherence Tomography: Mechanism and Clinical Applications," {\it University of Arizona}, 2015.}

%   \item {\bf R. Zhang} and C. Li, "Surface-Enhanced Raman Scattering Substrate Synthesis and Characterization", {\it Chongqing University}, 2013.
\end{itemize}




% \section*{References}
% \begin{itemize}
% \item Dr. Siyang Cao\\
% Department of Electrical and Computer Engineering\\
% {\it University of Arizona}\\
% Email: \href{mailto:caos@email.arizona.edu}{caos@email.arizona.edu}\\
% Phone: 520-621-4521
% \end{itemize}

\bigskip




\end{document}