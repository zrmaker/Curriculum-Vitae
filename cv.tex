\documentclass[letterpaper,10pt]{article}

\usepackage{tabularx}
\usepackage{hyperref}
\usepackage{geometry}
\usepackage{fancyhdr}
\pagestyle{fancy}
\usepackage[T1]{fontenc}
%\usepackage[sc,osf]{mathpazo}
\usepackage[sc]{mathpazo}
\usepackage{sectsty}
\PassOptionsToPackage{hyphens}{url}\usepackage{hyperref}

\def\name{Renyuan Zhang}

\hypersetup{
  colorlinks = true,
  urlcolor = black,
  pdfauthor = {\name},
  pdfkeywords = {Renyuan Zhang, Curriculum Vitae, Resume},
  pdftitle = {\name: Curriculum Vitae},
  pdfsubject = {Curriculum Vitae},
  %pdfpagemode = UseNone
}

\geometry{
%  body={6.5in, 9in},
  left=.75in,
  top=.75in,
  right=.75in,
  bottom=1in
}

\linespread{1.0}
\setlength{\parskip}{-5pt}
\setlength{\parindent}{0pt}

\urlstyle{rm}
\fancyhf{}
\rfoot{\thepage}
\pagestyle{fancy}
\renewcommand{\headrulewidth}{0pt}

\renewcommand{\familydefault}{\rmdefault}
\sectionfont{\rmfamily\mdseries\large}
\subsectionfont{\rmfamily\mdseries\itshape\small}
\subsubsectionfont{\rmfamily\mdseries\itshape}

% Other possible font commands include:
% \ttfamily for teletype,
% \sffamily for sans serif,
% \bfseries for bold,
% \scshape for small caps,
% \normalsize, \large, \Large, \LARGE sizes.

% Make lists without bullets
\renewenvironment{itemize}{
  \begin{list}{}{
    \setlength{\leftmargin}{1.5em}
  }
}{
  \end{list}
}


\begin{document}
{\huge \name \textnormal \ (Leo)}

% Alternatively, print name centered and bold:
%\centerline{\huge \bf \name}

\vspace{0.25in}

\begin{minipage}{0.55\linewidth}
  Research Assistant \\
  \href{http://ece.arizona.edu/}{Department of Electrical and Computer Engineering} \\
\href{http://www.arizona.edu/}{\it University of Arizona} \\
  Tucson, AZ 85721
\end{minipage}
\begin{minipage}{0.4\linewidth}
  \begin{tabular}{ll}
    Phone: & (520) 878-8630 \\
    Email: & \href{mailto:ryzhang@email.arizona.edu}{ryzhang@email.arizona.edu} \\
    Address: & 2724 N Neruda Ln, Tucson, AZ 85712\\
    %Homepage: & \href{http://www.stat-or.unc.edu/}{\tt http://www.stat-or.unc.edu/} \\
    GitHub: & \href{https://github.com/zrmaker}{zrmaker}
  \end{tabular}
\end{minipage}


%\section*{Personal}
%
%\begin{itemize}
%\item Born on September 29, 1895.
%\item United States Citizen.
%\end{itemize}

\section*{Summary of Qualifications}
\begin{itemize}
  \item 4+ years of research and engineering experience in the field of radar, automated driving, imaging and sensor networks.
  \item Intensive experience in programming in MATLAB, C/C++, Python, Java and engineering related languages in Windows and Linux environment.
  \item A strong self motivating ability and dedication to promoting effective teamwork. A strong ability to lead a research team.
\end{itemize}


\section*{Education}

\begin{itemize}
  \item Ph.D. in Electrical and Computer Engineering\hfill Aug. 2015 - Present\\ {\it University of Arizona}\\Research interest: radar signal processing, radar imaging, automotive radar, micro-doppler signatures.
  \item M.S. in Optical Sciences\hfill Aug. 2013 - Aug. 2015\\ {\it University of Arizona}\\Research interest: optical imaging, line CCD, optical coherence tomography.
  \item B.S. in Optoelectronic Engineering\hfill Sept. 2009 - June 2013\\ {\it Chongqing University}
\end{itemize}


\section*{Publications}

\subsection*{Journal Articles}

\begin{itemize}
\item \href{http://www.mdpi.com/1424-8220/17/6/1419}{{\bf R. Zhang} and S. Cao, "3D Imaging Millimeter Wave Circular Synthetic Aperture Radar," {\it Sensors}, vol. 17, no. 6, p. 1419, June 2017.}
\end{itemize}

\subsection*{Proceedings}

\begin{itemize}
\item {\bf R. Zhang} and S. Cao, "Support Vector Machines for Classification of Automotive Radar Interference," {\it 2018 IEEE Radar Conference (RadarConf)}, Oklahoma City, OK, USA, April 2018. (in press)
\item \href{http://ieeexplore.ieee.org/document/7944286/}{{\bf R. Zhang} and S. Cao, "Compressed Sensing For Portable Millimeter Wave 3D Imaging Radar," {\it 2017 IEEE Radar Conference (RadarConf)}, Seattle, WA, USA, May 2017, pp. 0663-0668.}
\item \href{http://ieeexplore.ieee.org/abstract/document/7944216/}{{\bf R. Zhang} and S. Cao, "Portable Millimeter Wave 3D Imaging Radar," {\it 2017 IEEE Radar Conference (RadarConf)}, Seattle, WA, USA, May 2017, pp. 0298-0303.}
\end{itemize}

\subsection*{Dissertation and Thesis}

\begin{itemize}
\item \href{https://scholar.google.com/citations?view_op=view_citation&hl=en&user=PpPf3BoAAAAJ&citation_for_view=PpPf3BoAAAAJ:9yKSN-GCB0IC}{{\bf R. Zhang} and K. Kieu, "Fiber Based Spectral Domain Optical Coherence Tomography: Mechanism and Clinical Applications," {\it University of Arizona}, 2015.}

\item {\bf R. Zhang} and C. Li, "Surface-Enhanced Raman Scattering Substrate Synthesis and Characterization", {\it Chongqing University}, 2013.
\end{itemize}


\section*{Professional Experience}
\begin{itemize}
\item [--] Research Assistant at Department of Electrical and Computer Engineering \hfill 2015 - Present\\
Advisor: Dr. Siyang Cao, {\it University of Arizona}.
\begin{itemize}
\item [--] Radar Interference Detection, Classification and Mitigation\\
Fields: Radar Interference (RFI), Machine Learning, DSP.\\
Working on simulation of 77 GHz automotive radar interference. Use machine learning methods to classify different interference range-doppler response results from PRI difference, long chirp, etc.. And use filter design, circular scanning array antenna and advanced signal processing methods to mitigate interference.
\item [--] Automotive Radar Measurement with 3D Printing Lens\\
With Min Liang and Jin-pil Tak.\\
Fields: Automotive Radar, DSP, RF and Antenna Theory, Antenna Measurements.\\
Measurements of real automobile with 77 GHz automotive radar and 3D priting lens (URL: \url{http://techlaunch.arizona.edu/news/startup-licenses-ua-invented-radar-system}).
\item [--] 3D Imaging Millimeter Wave Circular Synthetic Aperture Radar\\
Fields: SAR, DSP, Radar Imaging, mmWave Imaging, Compressed Sensing.\\
Paper published above.
\end{itemize}

\item [--] Sensor Engineer at TuSimple \hfill Sept. 2017 - Mar. 2018\\
{\it TuSimple LLC}, Tucson, AZ.
\begin{itemize}
\item [--] Autoliv\textsuperscript{\textregistered} 77 GHz multi-mode radar ROS driver development and evaluation.
\item [--] Bosch\textsuperscript{\textregistered} 77 GHz long-range radar and mid-range radar ROS driver development.
\item [--] Delphi\textsuperscript{\textregistered} 77 GHz electronic scanning radar evaluation.
\item [--] Hokuyo\textsuperscript{\textregistered} URG-04LX-UG01 Scanning Laser Rangefinder development and truck trailer monitor/filter project.
\item [--] Industrial radar signal filtering and target recognition development.

\end{itemize}

\item [--] Research Assistant of Nonlinear Optics at College of Optical Sciences \hfill 2014 - 2015\\
Advisor: Dr. Khanh Kieu, {\it University of Arizona}.
\begin{itemize}
\item [--] Thesis on Fiber Based Spectral Domain Optical Coherence Tomography: Mechanism and Clinical Applications\\
Fields: Optical Imaging, Interference, Lens Design, Spectral Domain Analysis, Fiber Optics, Medical Imaging, OCT.
\end{itemize}


\item [--] Research Assistant of Applied Optics at College of Optical Sciences \hfill 2013 - 2014\\
Advisor: Dr. Rongguang Liang, {\it University of Arizona}.
\begin{itemize}
\item [--] Confocal Microscopy\\
Fields: Optical Imaging, Spatial Pinhole, Lens Design.
\end{itemize}


\end{itemize}
\section*{Skills}
\begin{tabularx}{\textwidth}{lX}
{\bf Programming:} & Mathworks MATLAB, NI LabVIEW, Eclipse JAVA, Visual Studio C/C++/C\#, Python, R, CSS, HTML and Intel FPGA SDK.\\
{\bf RF \& EM:} & ANSYS EM suite and Keysight ADS.\\
{\bf Sensors:} & Radar, LiDAR, CMOS, CCD, sonar, depth sensor, microphone and Microsoft Kinect.\\
{\bf Machine Learning:} & SVM, ANN, RNN, {\it k}-NN, {\it k}-means, naive Bayes, decision tree and mixture model (Gaussian).\\
{\bf CAD \& Production:} & SOLIDWORKS, Autodesk AutoCAD and Adobe Creative Cloud (Photoshop, Illustrator, Premiere Pro).\\
{\bf Operating Systems:} & Windows and Ubuntu.\\
{\bf Embedded Systems:} & NI control and acquisition suites and Arduino.\\
{\bf Others:} & Digital signal processing (DSP), imaging processing, range-doppler processing, controller area network (CAN) for sensor fusion.\\
\end{tabularx}

\section*{References}
\begin{itemize}
\item Dr. Siyang Cao\\
Department of Electrical and Computer Engineering\\
{\it University of Arizona}\\
%1230 E. Speedway Blvd.\\
%P.O. Box 210104\\
%Tucson, AZ 85721-0104\\
Email: \href{mailto:caos@email.arizona.edu}{caos@email.arizona.edu}\\
Phone: 520-621-4521
\end{itemize}
\bigskip




\end{document}