\documentclass[letterpaper,9pt]{article}

\usepackage{tabularx}
\usepackage{hyperref}
\usepackage{geometry}
\usepackage{fancyhdr}
\usepackage[T1]{fontenc}
%\usepackage[sc,osf]{mathpazo}
\usepackage[sc]{mathpazo}
\usepackage{sectsty}

\pagestyle{fancy}
\PassOptionsToPackage{hyphens}{url}\usepackage{hyperref}

\def\name{Renyuan Zhang}

\hypersetup{
  colorlinks = true,
  urlcolor = black,
  pdfauthor = {\name},
  pdfkeywords = {Renyuan Zhang, Curriculum Vitae, Resume},
  pdftitle = {\name: Curriculum Vitae},
  pdfsubject = {Curriculum Vitae},
  %pdfpagemode = UseNone
}

\geometry{
%  body={6.5in, 9in},
  left=.75in,
  top=.75in,
  right=.75in,
  bottom=1in
}

\linespread{1.0}
\setlength{\parskip}{-5pt}
\setlength{\parindent}{0pt}

\urlstyle{rm}
\fancyhf{}
\rfoot{\thepage}
\pagestyle{fancy}
\renewcommand{\headrulewidth}{0pt}

\renewcommand{\familydefault}{\rmdefault}
\sectionfont{\rmfamily\mdseries\large}
\subsectionfont{\rmfamily\mdseries\itshape\small}
\subsubsectionfont{\rmfamily\mdseries\itshape}

% Other possible font commands include:
% \ttfamily for teletype,
% \sffamily for sans serif,
% \bfseries for bold,
% \scshape for small caps,
% \normalsize, \large, \Large, \LARGE sizes.

% Make lists without bullets
\renewenvironment{itemize}{
  \begin{list}{}{
    \setlength{\topsep}{0pt}
    \setlength{\itemsep}{0pt}
    \setlength{\parsep}{0pt}
    \setlength{\partopsep}{0pt}
    \setlength{\leftmargin}{1.5em}
  }
}{\end{list}}

\begin{document}
{\huge \name \textnormal \ (Leo), Ph.D.}

\vspace{1em}

\begin{minipage}{0.55\linewidth}
  Ph.D. in Electrical and Computer Engineering \\
  % Research Assistant \\
  % \href{http://ece.arizona.edu/}{Department of Electrical and Computer Engineering} \\
  \href{http://www.arizona.edu/}{\it University of Arizona} \\
  Tucson, AZ 85721
\end{minipage}
\begin{minipage}{0.4\linewidth}
  \begin{tabular}{ll}
    Phone: & (520) 878-8630 \\
    Email: & \href{mailto:ryzhang@email.arizona.edu}{ryzhang@email.arizona.edu} \\
    % Address: & 2724 N Neruda Ln, Tucson, AZ 85712\\
    LinkedIn: & \href{https://www.linkedin.com/in/zrmaker/}{https://www.linkedin.com/in/zrmaker/} \\
    GitHub: & \href{https://github.com/zrmaker}{zrmaker}
  \end{tabular}
\end{minipage}

%\section*{Personal}
%
%\begin{itemize}
%\item Born on November 30, 1990.
%\item Suzhou, China.
%\end{itemize}

\section*{Summary of Qualifications}
\begin{itemize}
  \setlength{\itemindent}{1em}
  % \setlength{\itemsep}{.25em}
  \item [$\bullet$] 4+ years of research and engineering experience in the field of radar, automated driving and imaging.
  \item [$\bullet$] Understanding of ADAS sensors such as radar, camera, sonar, GPS, IMU, and lidar.
  \item [$\bullet$] Intensive experience in programming in MATLAB, C/C++, Python, R and engineering related languages in Windows and Linux environment.
  \item [$\bullet$] A strong self motivating ability and dedication to promoting effective teamwork. A strong ability to lead a research team.
\end{itemize}


\section*{Skills}
\begin{tabularx}{\textwidth}{lX}
  {\bf Programming:} & Python, Mathworks\textsuperscript{\textregistered} MATLAB, R, NI LabVIEW, C/C++/C\#, JAVA.\\
  {\bf Sensors:} & Radar, LiDAR, CMOS camera, CCD camera, sonar,  microphone and Microsoft Kinect.\\
  {\bf Machine Learning:} & SVM, ANN, CNN, RNN, {\it k}-NN, {\it k}-means, naive Bayes, decision tree and mixture model (Gaussian).\\
  {\bf CAD \& Production:} & SOLIDWORKS, Autodesk AutoCAD and Adobe Creative Cloud (Photoshop, Illustrator, Premiere Pro).\\
  {\bf RF \& EM:} & ANSYS EM suite and Keysight ADS.\\
  {\bf Operating Systems:} & Windows and Ubuntu.\\
  {\bf Embedded Systems:} & NI control and acquisition suites and Arduino.\\
  {\bf Others:} & Digital signal processing (DSP), imaging processing, Nvidia\textsuperscript{\textregistered} CUDA, source control (git) and controller area network (CAN).\\
\end{tabularx}


\section*{Publications}

\subsection*{US Patents}

\begin{itemize}
  \item F. Deng, {\bf R. Zhang} and L. Nie, "Truck Trailer Angle Measurement using Single Beam Lidar," {\it US Patents}. (submitted 2018)

\end{itemize}

\subsection*{Journal Articles}

\begin{itemize}
  \item {\bf R. Zhang} and S. Cao, "Real-time Human Motion Behavior Detection via CNN using mmWave Radar," {\it IEEE Sensors Letters}. (submitted Sept. 2018)
  \item \href{http://www.mdpi.com/1424-8220/17/6/1419}{{\bf R. Zhang} and S. Cao, "3D Imaging Millimeter Wave Circular Synthetic Aperture Radar," {\it Sensors}, vol. 17, no. 6, p. 1419, June 2017.}
\end{itemize}

\subsection*{Proceedings}

\begin{itemize}
  \item {\bf R. Zhang} and S. Cao, "Robust and Adaptive Radar Elliptical Density-Based Spatial Clustering and Labelling for mmWave Radar Point Cloud Data," {\it 2019 IEEE Radar Conference}. (submitted Oct. 2018)
  \item \href{https://ieeexplore.ieee.org/abstract/document/8378586/}{{\bf R. Zhang} and S. Cao, "Support vector machines for classification of automotive radar interference," {\it 2018 IEEE Radar Conference (RadarConf18)}, Oklahoma City, OK, 2018, pp. 0366-0371.}
  \item \href{http://ieeexplore.ieee.org/document/7944286/}{{\bf R. Zhang} and S. Cao, "Compressed Sensing For Portable Millimeter Wave 3D Imaging Radar," {\it 2017 IEEE Radar Conference (RadarConf)}, Seattle, WA, USA, May 2017, pp. 0663-0668.}
  \item \href{http://ieeexplore.ieee.org/abstract/document/7944216/}{{\bf R. Zhang} and S. Cao, "Portable Millimeter Wave 3D Imaging Radar," {\it 2017 IEEE Radar Conference (RadarConf)}, Seattle, WA, USA, May 2017, pp. 0298-0303.}
\end{itemize}

\subsection*{Dissertation and Thesis}

\begin{itemize}
  \item \href{https://scholar.google.com/citations?view_op=view_citation&hl=en&user=PpPf3BoAAAAJ&citation_for_view=PpPf3BoAAAAJ:9yKSN-GCB0IC}{{\bf R. Zhang} and K. Kieu, "Fiber Based Spectral Domain Optical Coherence Tomography: Mechanism and Clinical Applications," {\it University of Arizona}, 2015.}

  \item {\bf R. Zhang} and C. Li, "Surface-Enhanced Raman Scattering Substrate Synthesis and Characterization", {\it Chongqing University}, 2013.
\end{itemize}


\section*{Professional Experience}
\begin{itemize}
  \setlength{\itemsep}{1em}
  \item Research Assistant at Department of Electrical and Computer Engineering \hfill 2015 - Present\\
Advisor: Dr. Siyang Cao, {\it University of Arizona}.
  \begin{itemize}
    \item [$\bullet$] Developing multi-target multi-input camera-radar fusion and classification.
    \item [$\bullet$] Researching non-synchronized incoherent MIMO radar angle resolution improvements.
    \item [$\bullet$] Researching radar point cloud machine learning method.
    \item [$\bullet$] Developing CUDA algorithms on radar signal processing.
    \item [$\bullet$] Researching on radar target clustering and classification.
    \item [$\bullet$] Achieved human behavior detection via CNN using micro-Doppler signatures by
  mmWave radar.
    % Fields: Micro-Doppler Signature, CNN.\\
    % A real-time, accurate and portable mmWave radar behavior detecting system is achieved.
    % The detection and classification is based on micro-Doppler signature of the radar targets. A convolution neural network (CNN) is implemented in the human behavior detection.
    \item [$\bullet$] Realized radar interference detection, classification and mitigation using SVM.
    % Fields: Radar Interference (RFI), Machine Learning, DSP.\\
    % Working on simulation of 77 GHz automotive radar interference. Use machine learning methods to classify different interference range-doppler response results from PRI difference, long chirp, etc.. And use filter design, circular scanning array antenna and advanced signal processing methods to mitigate interference.

    % \item [--] Automotive Radar Measurement with 3D Printing Lens\\
    % With Min Liang and Jin-pil Tak.\\
    % Fields: Automotive Radar, DSP, RF and Antenna Theory, Antenna Measurements.\\
    % Measurements of real automobile with 77 GHz automotive radar and 3D priting lens (URL: \url{http://techlaunch.arizona.edu/news/startup-licenses-ua-invented-radar-system}).
    \item [$\bullet$] Completed 3D imaging millimeter wave circular SAR.
    % Fields: SAR, DSP, Radar Imaging, mmWave Imaging, Compressed Sensing.\\
    % Paper published above.
  \end{itemize}

  \item Sensor Engineer at TuSimple \hfill Sept. 2017 - Mar. 2018\\
  {\it TuSimple LLC\textsuperscript{\textregistered}}, Tucson, AZ.
  \begin{itemize}
    \item [$\bullet$] Developed and evaluated Autoliv\textsuperscript{\textregistered} 77 GHz multi-mode radar ROS driver.
    \item [$\bullet$] Developed Bosch\textsuperscript{\textregistered} 77 GHz long-range radar and mid-range radar ROS driver.
    \item [$\bullet$] Evaluated Delphi\textsuperscript{\textregistered} 77 GHz electronic scanning radar.
    \item [$\bullet$] Finished Hokuyo\textsuperscript{\textregistered} URG-04LX-UG01 Scanning Laser Rangefinder development and truck trailer monitor/filter project.
    \item [$\bullet$] Written industrial radar signal filtering and target recognition.

  \end{itemize}

  \item Research Assistant of Nonlinear Optics at College of Optical Sciences \hfill 2014 - 2015\\
  Advisor: Dr. Khanh Kieu, {\it University of Arizona}.
  \begin{itemize}
    \item [$\bullet$] Developed and analyzed with hospitals using fiber based SD-OCT
    % Fields: Optical Imaging, Interference, Lens Design, Spectral Domain Analysis, Fiber Optics, Medical Imaging, OCT.
  \end{itemize}


  % \item Research Assistant of Applied Optics at College of Optical Sciences \hfill 2013 - 2014\\
  % Advisor: Dr. Rongguang Liang, {\it University of Arizona}.
  % \begin{itemize}
  % \item [$\bullet$] Confocal Microscopy
  % % Fields: Optical Imaging, Spatial Pinhole, Lens Design.
  % \end{itemize}


\end{itemize}



\section*{Education}

\begin{itemize}
  \setlength{\itemsep}{1em}
  \item Ph.D. in Electrical and Computer Engineering\hfill Aug. 2015 - Present\\ {\it University of Arizona}\\Research interest: radar signal processing, automotive radar, micro-doppler signatures, radar imaging, CUDA on radar signal processing, machine learning on human behaviros using radar.
  \item M.S. in Optical Sciences\hfill Aug. 2013 - Aug. 2015\\ {\it University of Arizona}\\Research interest: optical imaging, line CCD, optical coherence tomography.
  \item B.S. in Optoelectronic Engineering\hfill Sept. 2009 - June 2013\\ {\it Chongqing University}
  \item Udacity Self-Driving Car Nanodegree\hfill Oct. 2017 - Sep. 2018\\ {\it Udacity}
\end{itemize}


\section*{References}
\begin{itemize}
\item Dr. Siyang Cao\\
Department of Electrical and Computer Engineering\\
{\it University of Arizona}\\
%1230 E. Speedway Blvd.\\
%P.O. Box 210104\\
%Tucson, AZ 85721-0104\\
Email: \href{mailto:caos@email.arizona.edu}{caos@email.arizona.edu}\\
Phone: 520-621-4521
\end{itemize}

\bigskip




\end{document}